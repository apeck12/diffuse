\documentclass{article}
\usepackage{amssymb,amsfonts,amsmath,mathrsfs}
\usepackage[pdftex]{graphicx}
\usepackage{caption}
\usepackage[skip=0cm,list=true,labelfont=it]{subcaption}
\usepackage[top=2cm, bottom=1.5cm, right=2cm, left=2cm]{geometry}
\bibliographystyle{unsrt}
\usepackage{color}
\usepackage{braket}
\usepackage{sectsty}
\usepackage{fancyhdr}
\usepackage{tipa}
\usepackage{setspace}
\newcommand{\lambdabar}{\mbox{\textipa{\textcrlambda}}}

%\usepackage[numbers,sort&compress]{natbib}

\sectionfont{\fontsize{10}{10}\selectfont}
\subsectionfont{\fontsize{10}{10}\selectfont}
 
\pagestyle{plain}
\begin{document}

\section{Derivation of scattering intensities for different types of disorder}
\subsection{The ideal lattice} 
The intensity of scattered X-rays as a function of the wave vector, $\vec{q}$, from an ideal lattice can be described by: 
\begin{equation}
I(\vec{q}) = \vert \sum\limits_{n=1}^N f(\vec{q})e^{i\vec{q}\cdot\vec{r_n}}\vert^2
\end{equation}
where $\vec{r_n}$ is a unit lattice vector: $\vec{r_n}$ = $\vec{r_o}$ + $h$$\vec{r_x}$ + $k$$\vec{r_y}$ + $l$$\vec{r_z}$, with h, k, and l corresponding to the unit cell dimensions. The vector $\vec{r_o}$ can be taken to be the origin without any loss of generality. In one dimension, the above can be simplified to:
\begin{equation}
I(\vec{q}) = \vert f(\vec{q}) \vert^2   \vert \sum\limits_{h} e^{ih\theta}\vert^2
\end{equation}
by rewriting the dot product as an angle, $\theta$ = $\vec{q}$$\cdot$$\vec{r_x}$, where $\theta$ is real and $h$ is an integer. Consider the following two cases: when $\theta$ = 2$\pi$, the exponential term is unity and there is constructive interference that scales with the number of unit cells, N. When $\theta$ $\neq$ 2$\pi$, the exponent averages to 0, resulting in zero intensity due to destructive interference.

\subsection{Lattice with translational disorder} 
Assume that each unit cell is displaced from its mean lattice position, and that these displacements sample an independent and identically distributed (i.i.d.) Gaussian. In this case, a displacement term is added to the unit lattice vector: $\vec{r_h}$ = $h$$\vec{r_x}$ + $\vec{\Delta_h}$. The intensity of scattered X-rays is: 
\begin{eqnarray*}
I(\vec{q}) = \vert \sum\limits_{n=1}^N f(\vec{q})e^{i\vec{q}\cdot(h\vec{r_x} + \vec{\Delta_h})}\vert^2 
= \vert f(\vec{q}) \vert^2 \sum\limits_{h} e^{i\vec{q}\cdot(h\vec{r_x} + \vec{\Delta_h})} \sum\limits_{h'} e^{-i\vec{q}\cdot(h'\vec{r_x} + \vec{\Delta_{h'}})} \\
= \vert f(\vec{q}) \vert^2 \sum\limits_{h}\sum\limits_{h'} e^{i\vec{q}\cdot\vec{r_x}(h-h')+ i\vec{q}\cdot(\vec{\Delta_h} - \vec{\Delta_{h'}})} \\
= \vert f(\vec{q}) \vert^2  [ \sum\limits_{h}e^{0} + \sum\limits_{h}\sum\limits_{h', h'\neq h} e^{i\vec{q}\cdot\vec{r_x}(h-h')+ i\vec{q}\cdot(\vec{\Delta_h} - \vec{\Delta_{h'}})} ]
\end{eqnarray*}
Because $\vec{\Delta_h}$ and $\vec{\Delta_{h'}}$ are i.i.d. Gaussians with means of zero, their difference is also an i.i.d. Gaussian with a mean of zero and standard deviation of 2$\sigma$$^2$: $\vec{\Delta_{h, h'}}\sim N(0, \sigma^2)$, and $G_{h-h'} = (\vec{\Delta_{h}} - \vec{\Delta_{h'}}) \sim N(0, 2\sigma^2)$. Note that the characteristic function, $\phi(t)$, of a random variable is the Fourier transform of its probability density function. In the case of a normal distribution, the characteristic function evaluates to: 
\begin{equation}
\phi(t) = \int p(t) e^{it\cdot x} = e^{i\mu t}\cdot e^{\frac{-1}{2}\sigma^2 h^2}
\end{equation}
The above, along with substitution of $\vec{q}\cdot\vec{r_x} =  \theta$, can then be used to simplify the equation for the scattered intensity as follows:
\begin{eqnarray*}
I(\vec{q}) = \vert f(\vec{q}) \vert^2 [ N + \sum\limits_{h}\sum\limits_{h', h'\neq h} e^{i\vec{q}\cdot\vec{r_x}(h-h')+ i\vec{q} G_{hh'}}] 
= \vert f(\vec{q}) \vert^2 [ N + \sum\limits_{h}\sum\limits_{h', h'\neq h} e^{ih\theta} \int p(G_{hh'})e^{i\vec{q}G_{hh'}} dG_{hh'}] \\
= \vert f(\vec{q}) \vert^2 [ N + \sum\limits_{h}\sum\limits_{h', h'\neq h} e^{ih\theta} e^{-\sigma^2\vec{q}^2}]
= \vert f(\vec{q}) \vert^2 [ N + e^{-\sigma^2\vec{q}^2}\sum\limits_{h}\sum\limits_{h', h'\neq h} e^{ih\theta} ] \\
= \vert f(\vec{q}) \vert^2 [ N(1-e^{-\sigma^2\vec{q}^2}) + e^{-\sigma^2\vec{q}^2}\sum\limits_{h}\sum\limits_{h'} e^{ih\theta} ]
\end{eqnarray*}
The first term corresponds to the Fourier transform of a single unit cell and does not contain phase information. The second term is the Bragg component and scales with N$^2$ due to the double sum, in contrast to the first term which scales with N, so dominates at $\theta = 2\pi$ conditions. It is modulated by the $e^{-\sigma^2\vec{q}^2}$ term, which diminishes the Bragg intensities with increasing resolution and displacement.

\subsection{Lattice with configurational disorder}
Assume that each atom is displaced from its mean position. The intensity of a single unit cell can be expressed as:
\begin{equation}
I_{cell}(\vec{q}) = \langle \vert \sum\limits_{k} f_k(\vec{q})e^{i\vec{q}\cdot\vec{r_k}}\vert^2 \rangle
\end{equation}
where $f_k$ is the atomic form factor and $\vec{r_k} = \vec{\mu_k} + \vec{\delta_k}$.
Expanding the above,
\begin{equation}
I_{cell}(\vec{q}) = \sum\limits_{k} f_k(\vec{q})e^{i\vec{q}\cdot(\vec{r_k} + \vec{\delta_k})} \sum\limits_{k'} f_{k'}(\vec{q})e^{-i\vec{q}\cdot(\vec{r_{k'}} + \vec{\delta_{k'}})}
= \sum\limits_{k}\sum\limits_{k'}f_{k}f_{k'}e^{i\vec{q}[(\vec{r_{k}} - \vec{r_{k'}}) + (\vec{\delta_{k}} - \vec{\delta_{k'}})]}
\end{equation}
Let $x_0 = \vec{r_k} - \vec{r_k'}$, $x = \vec{\delta_{k}} - \vec{\delta_{k'}}$, and P(x) be a multivariate Gaussian model for the atomic positional probabilities. Then the exponential term becomes:
\begin{equation}
\int dx P(x) e^{i\vec{q^T}(\vec{x_0} + \vec{x})} = e^{i\vec{q^T} x_0} \int dx \frac{e^{\frac{-1}{2}\vec{x^T}V^{-1}\vec{x}} \cdot e^{i\vec{q^T}\vec{x}}} {\int dx e^{\frac{1}{2} \vec{x^T}V^{-1}\vec{x}}} = e^{i\vec{q^T}\vec{x_0}} \cdot Z^{-1} \int \frac{dx \cdot e^{ \frac{-1}{2}\vec{x^T}V^{-1}\vec{x} + i\vec{q^T}\vec{x}}}{Z^{-1} \cdot e^{- \frac{1}{2} \vec{q^T}V\vec{q}}}
\end{equation}
where $Z = \int dx e^{\frac{1}{2} x^Tvx}$ is a normalization constant integrated over all of q-space. Because $x^Tv$ is symmetric and positive matrix, the expression in the integrand can be simplified to $e^{- \frac{1}{2} \vec{q^T}V_{kk'}\vec{q}}$, where $V_{kk'}=\langle \vec{x}\vec{x^T} \rangle$, by the definition of an n-dimensional Gaussian integral. Thus,
\begin{equation}
I_{ensemble} (\vec{q}) = \sum\limits_{k}\sum\limits_{k'}f_{k}f_{k'}e^{i\vec{q}(\vec{r_{k}} - \vec{r_{k'}})}e^{-\frac{1}{2}\vec{q^T}V_{kk'}\vec{q}}
\end{equation}
$V_{kk'}$ is the interatomic displacement covariance matrix between atoms k and k':
\begin{equation}
V_{kk'} = \langle (\delta_k - \delta_k'^T)(\delta_k - \delta_k'^T) \rangle = \langle \delta_k \delta_k^T \rangle + \langle \delta_k' \delta_k'^T \rangle - 2 \langle \delta_k \delta_k'^T \rangle
\end{equation}
where each term in the expansion is a 3 x 3 symmetric covariance matrix of interatomic displacements. Thus, $I(q)$ can be rewritten as follows:
\begin{equation}
I_{ensemble} (\vec{q}) = \sum\limits_{k}\sum\limits_{k'}f_{k}e^{-W_k}f_{k'}e^{-W_k'}e^{i\vec{q}(\vec{r_{k}} - \vec{r_{k'}})}e^{\vec{q^T} \langle \delta_k \delta_k'^T \rangle \vec{q}}
\end{equation}
where $W_k = \frac{1}{2}\vec{q^T} \langle \delta_k \delta_k^T \rangle \vec{q}$ is the Debye-Waller factor.
\newline
\newline In the case of a model that only has sufficient resolution to account for isotropic disorder, eq. 7 can be rewritten by relating $C_{kk'}$, the N x N correlation matrix (where N is the number of atoms) to anisotropic displacement covariance matrix, $V$ as follows:
\begin{equation}
C_{kk'} = \frac{Tr(\langle \delta_k \delta_k'^T \rangle)}{(Tr(\langle \delta_k \delta_k^T \rangle)Tr(\langle \delta_k' \delta_k'^T \rangle))^{\frac{1}{2}}} = \frac{ \langle \delta_k \delta_{k'}^T \rangle}{B_k^{\frac{1}{2}}B_{k'}^{\frac{1}{2}}}
\end{equation}
where $B_k = \langle \delta_k \delta_k^T \rangle$, and thus related to the B-factor. Then,
\begin{equation}
I_{ensemble} (\vec{q}) = \sum\limits_{k}\sum\limits_{k'} f_{k}f_{k'}e^{i\vec{q}(\vec{r_{k}} - \vec{r_{k'}})}e^{-\frac{1}{2}\vec{q}^2(B_k + B_{k'} - 2 C_{kk'}(B_k B_{k'})^\frac{1}{2})}
\end{equation}


\subsection{Inference of the Correlation Matrix}

Let us assume that we have many measurements of $I_\mathrm{ensemble}$ and wish to infer a model of the form just discussed. To do this, we may minimize the least squares loss function
\[
\mathcal{O} ( \{ C_{kk'} \} )= \sum_{\{\mathbf{q}\}} \left| I_\mathrm{esb}^\mathrm{obs} (\mathbf{q}) 
- I_\mathrm{esb}^\mathrm{calc} (\mathbf{q}; \{ C_{kk'} \}) \right|^2
\]
where ``obs'' and ``calc'' indicate the observed and modeled values respectively. Our task is to vary $ \{ C_{kk'} \}$ to minimize $\mathcal{O}$. 

Let $A_{k k'} (\mathbf{q}) \equiv f_{k}f_{k'} \exp \{ \, i \mathbf{q} (\mathbf{r_{k}}- \mathbf{r_{k'}}) - \frac{1}{2} q^2(B_k + B_{k'} ) \, \}$ and $\alpha_{k k'} \equiv (B_k B_{k'})^\frac{1}{2}$. Then we may write
\[
I_\mathrm{esb}^\mathrm{calc} (\mathbf{q}; \{ C_{kk'} \}) = \sum_{k k'} A_{k k'} (\mathbf{q}) \, e^{\alpha_{k k'} C_{k k'}}
\]
and the derivatives of the objective function take on a simple form
\begin{align*}
\frac{\partial \mathcal{O}} {\partial C_{kk'}} &= \sum_{\{\mathbf{q}\}}
2 \alpha_{k k'} A_{k k'} e^{\alpha_{k k'} C_{k k'}} 
\left( A_{k k'} e^{\alpha_{k k'} C_{k k'}} - I_\mathrm{esb}^\mathrm{obs}  \right) \\
%
&= \sum_{\{\mathbf{q}\}} 2 \alpha_{k k'} I_\mathrm{esb}^\mathrm{calc} \left[
I_\mathrm{esb}^\mathrm{calc} - I_\mathrm{esb}^\mathrm{obs}
\right]
\end{align*}
this shows the derivative is zero and the objective is at an extreme when $C_{kk'} = \hat{C}_{kk'}$ with
\[
\hat{C}_{kk'} = \alpha_{kk'}^{-1} 
\log \frac{ \sum_{\{\mathbf{q}\}} I_\mathrm{esb}^\mathrm{obs} (\mathbf{q}) A_{kk'} (\mathbf{q})}
{\sum_{\{\mathbf{q}\}} A_{kk'}^2 (\mathbf{q}) }
\]
computing the second derivative
\[
\frac{\partial^2 \mathcal{O}} {\partial C_{kk'}^2} = \sum_{\{\mathbf{q}\}}
2 \alpha_{k k'}^2 A_{k k'} e^{\alpha_{k k'} C_{k k'}} 
\left( 2 A_{k k'} e^{\alpha_{k k'} C_{k k'}} - I_\mathrm{esb}^\mathrm{obs} \right)
\]
and substituting $C = \hat{C}_{kk'}$
\[
\frac{\partial^2 \mathcal{O}} {\partial C_{kk'}^2} (\hat{C}_{kk'}) =  2 \alpha_{k k'}^2 
\frac{\sum_{\{\mathbf{q}\}} A_{kk'}^2 (\mathbf{q})I_\mathrm{esb}^{\mathrm{obs}\, 2}(\mathbf{q})}
{\sum_{\{\mathbf{q}\}} A_{kk'}^2 (\mathbf{q})}
\]
proves that this extremum is indeed the minimum and $\mathcal{O}$ is convex.

\end{document}


