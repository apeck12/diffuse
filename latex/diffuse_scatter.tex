\documentclass{article}
\usepackage{amssymb,amsfonts,amsmath,mathrsfs}
\usepackage[pdftex]{graphicx}
\usepackage{caption}
\usepackage[skip=0cm,list=true,labelfont=it]{subcaption}
\usepackage[top=2cm, bottom=1.5cm, right=2cm, left=2cm]{geometry}
\bibliographystyle{unsrt}
\usepackage{color}
\usepackage{braket}
\usepackage{sectsty}
\usepackage{fancyhdr}
\usepackage{tipa}
\usepackage{setspace}
\newcommand{\lambdabar}{\mbox{\textipa{\textcrlambda}}}

%\usepackage[numbers,sort&compress]{natbib}

\sectionfont{\fontsize{10}{10}\selectfont}
\subsectionfont{\fontsize{10}{10}\selectfont}
 
\pagestyle{plain}
\begin{document}


Consider the scattering of x-rays from a crystal in the far-field. A general form of the scattered intensity, $I$, at some wavevector $\mathbf{q}$ in reciprocal space is:
\begin{equation}
I(\mathbf{q}) = \left| \sum\limits_{c} e^{-i \mathbf{q} \cdot \mathbf{u_c} } \sum\limits_{i} f_i e^{-i \mathbf{q} \cdot \mathbf{r}_i} e^{-i \mathbf{q} \cdot \boldsymbol{\delta_{c_i}}} \right| ^2
\end{equation}
where $\mathbf{u_c}$ is the vector from the origin of the coordinate system to the origin of unit cell $c$, $f_i$ is the atomic form factor of atom $i$, $r_i$ is the vector that describes the mean position of atom i relative to origin of the unit cell, and $\boldsymbol{\delta}_{c_i}(t)$ is the instantaneous displacement vector of atom i in unit cell $c$.

Adopting the compact notation $x_{ij} = x_i - x_j$ for $\mathbf{u}$, $\mathbf{r}$, and $\boldsymbol{\delta}$, we may write the time (or, equivalently, ensemble) averaged scattering
\begin{equation}
\langle I (\mathbf{q}) \rangle = \left\langle 
\sum\limits_{c,d} e^{-i \mathbf{q} \cdot \mathbf{u}_{c d}} 
\sum\limits_{i,j} f_i f_j e^{-i \mathbf{q} \cdot \mathbf{r}_{ij}} 
e^{-i \mathbf{q} \cdot \boldsymbol{\delta}_{c_i d_j}} 
\right \rangle
\label{eq:avgscatter}
\end{equation}
%
We make three critical assumptions about the structure of the statistical ensemble of atomic displacements $\boldsymbol{\delta}$:
%
% ASSUMPTIONS
%
\begin{enumerate}

\item The correlation range of atomic displacements is small with respect to the size of the crystal. Equivalently, the correlation between atomic displacements in different unit cells are independent,
\[
\langle \boldsymbol{\delta}_{c_i}^T \boldsymbol{\delta}_{d_j} \rangle = \mathbf{0} 
\ \mathrm{if} \ c \neq d
\]
and therefore we may write the displacements simply using their atomic index, for instance ${\delta}_{i}$. Note that if correlations exist between what might traditionally be considered a minimal unit cell, and one wishes to consider these correlations, it is possible to simply define a larger unit cell that encompasses the entire correlated region.

\item Atoms in different unit cells behave identically in a statistical fashion, so $p(\boldsymbol{\delta}_{c i}) = p(\boldsymbol{\delta}_{d i})$ for all $i$.

\item The atom displacements may be described by a pairwise multivariate normal distribution, with zero mean and covariance matrix $V_{ij} = \langle \boldsymbol{\delta}_{i}^T\boldsymbol{\delta}_{j} \rangle \in \mathcal{R}^{3 \times 3}$
\[
p( \boldsymbol{\delta}_{i}, \boldsymbol{\delta}_{j} ) \sim \mathrm{MVN}( \mathbf{0}, V_{ij})
\]
this is the simplest model that takes into account anisotropic correlations between atoms.

\end{enumerate}
%
%
Since the average is over pairwise probability distrubtions, we may re-write eq.~\ref{eq:avgscatter} as
\begin{align}
\langle I (\mathbf{q}) \rangle =
\sum\limits_{c,d} e^{-i \mathbf{q} \cdot \mathbf{u}_{c d}} 
\sum\limits_{i,j} f_i f_j e^{-i \mathbf{q} \cdot \mathbf{r}_{ij}} 
%
\iint p( \boldsymbol{\delta}_{c_i},  \boldsymbol{\delta}_{d_j} )
e^{-i \mathbf{q} \cdot ( \boldsymbol{\delta}_{c_i}  - \boldsymbol{\delta}_{d_j})} 
d \boldsymbol{\delta}_{c_i}  \, d \boldsymbol{\delta}_{d_j}
%
\end{align}
in appendix A1 we show 
\begin{equation}
\iint p( \boldsymbol{\delta}_{c_i},  \boldsymbol{\delta}_{d_j} )
e^{-i \mathbf{q} \cdot ( \boldsymbol{\delta}_{c_i}  - \boldsymbol{\delta}_{d_j})} 
d \boldsymbol{\delta}_{c_i}  \, d \boldsymbol{\delta}_{d_j}
=
\exp \left\{
- \frac{1}{2} \mathbf{q}^T V_{c_i c_i} \mathbf{q}
- \frac{1}{2} \mathbf{q}^T V_{d_j d_j} \mathbf{q}
+ \mathbf{q}^T V_{c_i d_j} \mathbf{q}
\right\}
\label{eq:disorder_term}
\end{equation}
under our assumptions, $V_{c_i d_j} = \mathbf{0}$ if $c \neq d$, and $V_{c_i c_i}$ is identical for all $c$, such that we may write $V_{c_i c_i} = V_{ii}$ (and similarly $V_{d_j d_j} = V_{jj}$). Thus, it is natural to split eq.~\ref{eq:avgscatter} into two parts -- one expressing interference \textit{between} unit cells (where $V_{c_i d_j} = \mathbf{0}$), and one expressing interference \textit{within} repeats of a single cell,
\begin{align}
\langle I (\mathbf{q}) \rangle =&
%
% between (incomplete)
\sum\limits_{c,d \neq c} e^{-i \mathbf{q} \cdot \mathbf{u}_{c d}} 
\sum\limits_{i,j} f_i f_j e^{-i \mathbf{q} \cdot \mathbf{r}_{ij}} 
%
e^{
- \frac{1}{2} \mathbf{q}^T V_{ii} \mathbf{q}
- \frac{1}{2} \mathbf{q}^T V_{jj} \mathbf{q}
} \\
%
% within (incomplete)
&+
N \sum\limits_{i,j} f_i f_j e^{-i \mathbf{q} \cdot \mathbf{r}_{ij}} 
%
e^{
- \frac{1}{2} \mathbf{q}^T V_{ii} \mathbf{q}
- \frac{1}{2} \mathbf{q}^T V_{jj} \mathbf{q}
+ \mathbf{q}^T V_{ij} \mathbf{q}
}  \\
%
=&
%
% between (complete)
\sum\limits_{c, d} e^{-i \mathbf{q} \cdot \mathbf{u}_{c d}} 
\sum\limits_{i,j} f_i f_j e^{-i \mathbf{q} \cdot \mathbf{r}_{ij}} 
%
e^{
- \frac{1}{2} \mathbf{q}^T V_{ii} \mathbf{q}
- \frac{1}{2} \mathbf{q}^T V_{jj} \mathbf{q}
} \\
%
% within (complete)
&+
N \sum\limits_{i,j} f_i f_j e^{-i \mathbf{q} \cdot \mathbf{r}_{ij}} 
%
e^{
- \frac{1}{2} \mathbf{q}^T V_{ii} \mathbf{q}
- \frac{1}{2} \mathbf{q}^T V_{jj} \mathbf{q}
}
%
\left[ 
e^{\mathbf{q}^T V_{ij} \mathbf{q}} - 1
\right]
%
\end{align}
with $N$ being the number of unit cells.

The astute reader will notice the first term as the traditional crystallographic scattering
\begin{align}
I(\mathbf{q})_{\mathrm{Bragg}} &=
\sum\limits_{c, d} e^{-i \mathbf{q} \cdot \mathbf{u}_{c d}} 
\sum\limits_{i,j} f_i f_j e^{-i \mathbf{q} \cdot \mathbf{r}_{ij}} 
e^{
- \frac{1}{2} \mathbf{q}^T V_{ii} \mathbf{q}
- \frac{1}{2} \mathbf{q}^T V_{jj} \mathbf{q}
} \\
%
&= \left| \left( 
\sum_c e^{-i \mathbf{q} \cdot \mathbf{u}_{c}} 
\right) \left(
\sum\limits_{i} f_i e^{-i \mathbf{q} \cdot \mathbf{r}_{i}} 
e^{
- \frac{1}{2} \mathbf{q}^T V_{ii} \mathbf{q}
} \right) \right|^2\\
%
&= \left| S( \mathbf{q} ) \right|^2  \left| F( \mathbf{q} ) \right|^2
\end{align}
where $V_{ii}$ is an anisotropic B-factor (also called the Debye Waller factor). $\left| S( \mathbf{q} ) \right|^2$ becomes a Dirac comb as the number of unit cells grows, showing this scattering is localized to discrete regions of $\mathbf{q}$.

The remaining scattering $\langle I (\mathbf{q}) \rangle - I(\mathbf{q})_{\mathrm{Bragg}}$ is typically termed the diffuse scattering
\[
I_\mathrm{diffuse} (\mathbf{q}) = N \sum\limits_{i,j} f_i f_j e^{-i \mathbf{q} \cdot \mathbf{r}_{ij}} 
%
e^{
- \frac{1}{2} \mathbf{q}^T V_{ii} \mathbf{q}
- \frac{1}{2} \mathbf{q}^T V_{jj} \mathbf{q}
}
%
\left[ 
e^{\mathbf{q}^T V_{ij} \mathbf{q}} - 1
\right]
\]
There are two notable features of the diffuse scattering. First, lacking the lattice transform $\left| S( \mathbf{q} ) \right|^2$ it is \textit{not} localized in reciprocal space. Second, it is non-trivial only if there is correlated displacements between the atoms $V_{ij} \neq \mathbf{0}$.

% ---------------------------------------------------
\section{Diffuse Scatter in Special Cases}

%%%
\subsection{No Cross-Correlation}
In the case $V_{ij}$ is $\mathbf{0}$ for all $\{i, j\}$, then all $i \neq j$ terms vanish, and
\[
I_\mathrm{diffuse} (\mathbf{q}) = N \sum_{i} f_i^2
%
\left[ 
1 - e^{- \mathbf{q}^T V_{ii} \mathbf{q}}
\right]
\]
just a relatively unstructured background remains, proportional to the anisotropic B-factors. Note anisotropy is still possible if the individual atomic displacements are highly anisotropic. The diffuse scatter will, however, lack the ``speckle'' features that normally distinguish it from background scattering.

%%%
\subsection{Correlated Bodies}
In the case that an entire set of atoms moves as a unit, the covariance between all pairs of atoms $V_{ij}$ can be written as a scalar $\sigma^2$, and the diffuse scatter becomes
\begin{align}
I_\mathrm{diffuse} (\mathbf{q}) =&
N \sum\limits_{i,j} f_i f_j e^{-i \mathbf{q} \cdot \mathbf{r}_{ij}} 
\left[ 1 - e^{- q^2 \sigma^2} \right] 
\nonumber \\ 
%
=& N \left[ 1 - e^{- q^2 \sigma^2} \right] |F(q)|^2
\end{align}
The diffuse scatter in this case is a scaled version of the molecular transform intensity. This expression was derived by Moore and later Chapman.


%%%
\subsection{Isotropic Real Space Correlation Length}
Here, we reproduce the isotropic correlation model of Clarage. Our approach is equivalent but more direct, employing the expression for the scattered intensity rather than reasoning about the Patterson. Clarage makes the following assumptions about the atomic correlations:
%
\begin{enumerate}

\item Isotropic, distance dependent interatomic correlation: 
$V_{ij} = \sigma^2 \Gamma(\mathbf{r}_{ij}) $ with $\Gamma$ a kernel describing the correlation length.

\item Isotropic B-factors: $V_{ii} = \sigma^2$ for all $i$.

\end{enumerate}

Using these simplifying approximations, we may write the diffuse scatter
\[
I_\mathrm{diffuse} (\mathbf{q}) = N \sum\limits_{i,j} f_i f_j e^{-i \mathbf{q} \cdot \mathbf{r}_{ij}} 
%
e^{- q^2 \sigma^2}
%
\left[ 
e^{q^2 \sigma^2 \Gamma(\mathbf{r}_{ij})} - 1
\right]
\]
Clarage suggests that so long as $q^2 \sigma^2 \Gamma(\mathbf{r}_{ij})$ is small, we may Taylor expand the exponential $\exp \{ q^2 \sigma^2 \Gamma(\mathbf{r}_{ij}) \} \approx 1 - q^2 \sigma^2 \Gamma(\mathbf{r}_{ij})$ and further simplify
\[
I_\mathrm{diffuse} (\mathbf{q}) = q^2 \sigma^2 \, e^{- q^2 \sigma^2} N 
\sum\limits_{i,j} f_i f_j e^{-i \mathbf{q} \cdot \mathbf{r}_{ij}} \Gamma(\mathbf{r}_{ij})
\]
For this approximation to hold, $q^2 \sigma^2 \Gamma(\mathbf{r}_{ij}) \ll 1$. Since $q \sim 1 \AA^{-1}$, $\sigma \ll 1 \AA$, and $\Gamma(\mathbf{r}_{ij}) \leq 1$ this is not a bad approximation. To assist in the analysis of this function, define a new kernel function
\[
s( \mathbf{q}, \mathbf{r}) = \sum_{i,j} f_i f_j \delta( \mathbf{r} - \mathbf{r}_{ij})
\]
with $\delta(\cdot)$ the Dirac delta (not a displacement). Using this kernel, we may re-write the diffuse scattering as a Fourier transform, rather than a discrete sum
\[
I_\mathrm{diffuse} (\mathbf{q}) = q^2 \sigma^2 \, e^{- q^2 \sigma^2} N 
\int s( \mathbf{q}, \mathbf{r})  \Gamma(\mathbf{r}) e^{-i \mathbf{q} \cdot \mathbf{r}} \, d \mathbf{r}
\]
using the Fourier convolution theorem (specifically, here that $ \mathcal{F}[ f \cdot g ] = \mathcal{F}^{-1}[f] \ast \mathcal{F}^{-1}[g]$, with $\ast$ denoting convolution), we can split the Fourier integral
\[
I_\mathrm{diffuse} (\mathbf{q}) = q^2 \sigma^2 \, e^{- q^2 \sigma^2} \frac{N}{16 \pi^6} 
\left[ \left( \int s( \mathbf{q}, \mathbf{r})  e^{i \mathbf{q} \cdot \mathbf{r}} \, d \mathbf{r} \right) \ast
\left( \int \Gamma(\mathbf{r})   e^{i \mathbf{q} \cdot \mathbf{r}} \, d \mathbf{r} \right) \right]
\]
let $\tilde{\Gamma} (\mathbf{q}) = \int \Gamma(\mathbf{r}_{ij})   e^{i \mathbf{q} \cdot \mathbf{r}_{ij}} \, d \mathbf{r}$ and notice 
\[
\int s( \mathbf{q}, \mathbf{r})  e^{i \mathbf{q} \cdot \mathbf{r}} \, d \mathbf{r} =
%
\int \sum_{i,j} f_i f_j \delta( \mathbf{r} - \mathbf{r}_{ij}) e^{i \mathbf{q} \cdot \mathbf{r}}  \, d \mathbf{r} =
%
\sum_{i,j} f_i f_j e^{i \mathbf{q} \cdot \mathbf{r_{ij}}}
\]
is the scattered intensity of a (fully ordered) single unit cell. We will call this $I_0(\mathbf{q})$. In summary, we may write the diffuse scatter as a convolution between this intensity function and the Fourier transform of the correlation kernel $\Gamma$,
\begin{equation}
I_\mathrm{diffuse} (\mathbf{q}) = q^2 \sigma^2 \, e^{- q^2 \sigma^2} \frac{N}{16 \pi^6} 
\left[ I_0(\mathbf{q}) \ast \tilde{\Gamma}(\mathbf{q}) \right]
\end{equation}


\section{Appendix A1}
We prove
\begin{equation}
\iint p( \boldsymbol{\delta}_{c_i},  \boldsymbol{\delta}_{d_j} )
e^{-i \mathbf{q} \cdot ( \boldsymbol{\delta}_{c_i}  - \boldsymbol{\delta}_{d_j})} 
d \boldsymbol{\delta}_{c_i}  \, d \boldsymbol{\delta}_{d_j}
=
\exp \left\{
- \frac{1}{2} \mathbf{q}^T V_{c_i c_i} \mathbf{q}
- \frac{1}{2} \mathbf{q}^T V_{d_j d_j} \mathbf{q}
+ \mathbf{q}^T V_{c_i d_j} \mathbf{q}
\right\}
\label{eq:disorder_term2}
\end{equation}
First, recall the characteristic function of a multivariate normal distribution is
\begin{equation}
\int \cdots \int f_{\boldsymbol{\mu}, \Sigma}(\mathbf{x}) e^{-i \mathbf{u} \cdot \mathbf{x}} \, d \mathbf{x}
= \exp \left\{ i \mathbf{u} \cdot \boldsymbol{\mu} - \frac{1}{2} \mathbf{u}^T \Sigma \mathbf{u} \right\}
\label{eq:mvnlemma}
\end{equation}
this can be shown by performing the Gaussian integral on the lhs directly.

We will write (\ref{eq:disorder_term2}) in canonical MVN form and then employ (\ref{eq:mvnlemma}). To do this, introduce the augmented variables
\[
\mathbf{x} = 
\begin{bmatrix}
\boldsymbol{\delta}_{c_i} \\
- \boldsymbol{\delta}_{d_j}
\end{bmatrix}
%
\ \ \mathrm{and} \ \
%
\mathbf{q}_2 = 
\begin{bmatrix}
\mathbf{q} \\
\mathbf{q}
\end{bmatrix}
\]
which are vectors in $\mathcal{R}^6$. The covariance matrix of $\mathbf{x}$ may be written in block form
\[
\Sigma = 
\langle \mathbf{x}^T  \mathbf{x} \rangle = 
\begin{bmatrix}
V_{{c_i}{c_i}} & -V_{{c_i}{d_j}} \\
-V_{{d_j}{c_i}} & V_{{d_j}{d_j}} \\
\end{bmatrix}
\]
simply using the definition $V_{ab} = \langle \boldsymbol{\delta}_{a}^T\boldsymbol{\delta}_{b} \rangle$. Recall also that $\langle \mathbf{x} \rangle = \mathbf{0}$. Now, we may write (\ref{eq:disorder_term2}) in terms of $\mathbf{x}$ and $\mathbf{q}_2$,
\[
= \int p(\mathbf{x} ) e^{- i \mathbf{q}_2 \cdot \mathbf{x} } \, d \mathbf{x}
\]
employing (\ref{eq:mvnlemma}) and simplifying completes the proof.

\bibliography{diffuse_scatter}

\end{document}


