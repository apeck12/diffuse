\documentclass{article}
\usepackage{amssymb,amsfonts,amsmath,mathrsfs}
\usepackage[pdftex]{graphicx}
\usepackage{caption}
\usepackage[skip=0cm,list=true,labelfont=it]{subcaption}
\usepackage[top=2cm, bottom=1.5cm, right=2cm, left=2cm]{geometry}
\bibliographystyle{unsrt}
\usepackage{color}
\usepackage{braket}
\usepackage{sectsty}
\usepackage{fancyhdr}
\usepackage{tipa}
\usepackage{setspace}
\usepackage{ulem}
\newcommand{\lambdabar}{\mbox{\textipa{\textcrlambda}}}

%\usepackage[numbers,sort&compress]{natbib}

\sectionfont{\fontsize{10}{10}\selectfont}
\sectionfont{\fontsize{10}{10}\selectfont}
 
\pagestyle{plain}

\title{Derivation of scattering intensities for macromolecular crystals}

\begin{document}

\maketitle

\tableofcontents

\newpage
\section{General expressions for Bragg and diffuse intensities}
The most general form of the scattered intensity, $I$, at some wavevector $\vec{q}$ in reciprocal space is \textcolor{red}{\cite{pmid19836331}}:
\begin{equation}
I(\vec{q}) = \vert \sum\limits_{c} e^{-i \vec{q} \cdot \vec{u_c} } \sum\limits_{i} f_i e^{-i \vec{q} \cdot \vec{r_i}} e^{-i \vec{q} \cdot \vec{\delta_{ci}}(t)} \vert ^2
\end{equation}
where $\vec{u_c}$ is the vector from the origin of the coordinate system to the origin of unit cell c, $f_i$ is the atomic form factor of atom i, $r_i$ is the vector that describes the mean position of atom i relative to origin of the unit cell, and $\vec{\delta_{ci}}(t)$ is the instantaneous displacement vector of atom i in unit cell c. The \sout{Bragg component of} \textcolor{red}{total intensity scattered} $I_{total}(\vec{q})$ is the average of the former over time \sout{determined by the average atomic positions} and can be written as follows:
\begin{equation}
I(\vec{q})_{total} = \sum\limits_{c,c'} e^{-i \vec{q} \cdot (\vec{u_c} - \vec{u_c'})} \sum\limits_{i,j} f_i f_j e^{-i \vec{q} (\vec{r_i} - \vec{r_j})} \langle e^{-i \vec{q} \cdot (\vec{\delta_{ci}}(t) - \vec{\delta_{c'j}}(t))} \rangle
\end{equation}
A probability model, $P$, for the displacement distribution of atom i must satisfy the following constraint:
\begin{equation}
\langle \vec{\delta_{i}} \rangle = \int P(\vec{\delta_{i}})\, \vec{\delta_{i}} \, d\vec{\delta_{i}} = 0
\end{equation}
where the integration is over all unit cells. Thus, a multivariate Gaussian would suffice: for example, $P(\vec{\delta_{i}}) =e^{-\frac{1}{2} \vec{x}^\intercal V \vec{x}}$, where $V = \langle \vec{\delta_{i}} \, \vec{\delta_{i}}^\intercal \rangle$. Here it is assumed that $V$ is identical when evaluated for equivalent atoms in different unit cells: $V_{ci,ci} = V_{c'i,c'i}$.
Then, the exponential decay term in the equation for $I(\vec{q})_{Bragg}$ can be evaluated as\footnotemark:
\begin{equation}
\langle e^{-i {\vec{q}} \cdot \vec{\delta}_{ci} } \rangle = \int P(\vec{\delta}_{ci})\, e^{-i \vec{\delta}_{ci} \cdot {\vec{q}}} \, d\vec{\delta}_{i,c} = e^{-\frac{1}{2} \vec{q}^\intercal V_{ci,ci} \vec{q} }
\end{equation}
and the equation for $I(\vec{q})_{Bragg}$ becomes:
\begin{equation}
\begin{aligned}
I(\vec{q})_{Bragg} & = \sum\limits_{c,c'} e^{-i \vec{q} \cdot (\vec{u_c} - \vec{u_c'})} \sum\limits_{i,j} f_i f_j e^{-i \vec{q} (\vec{r_i} - \vec{r_j})} e^{-\frac{1}{2} \vec{q}^\intercal V_{ci,ci} \vec{q}} e^{-\frac{1}{2} \vec{q}^\intercal V_{c'j,c'j} \vec{q}} \\
& = \sum\limits_{c,c'} e^{-i \vec{q} \cdot (\vec{u_c} - \vec{u_c'})} \sum\limits_{i,j} f_i f_j e^{-i \vec{q} (\vec{r_i} - \vec{r_j})} e^{-\frac{1}{2} (\langle (\vec{\delta_i} \cdot \vec{q})^2 \rangle + \langle (\vec{\delta_j} \cdot \vec{q})^2 \rangle) } \\
\end{aligned}
\end{equation}
where the final exponential term corresponds to the Debye-Waller or conventional B factors. The first sum over unit cells takes the form of the Dirac comb in the limit that the number of unit cells, $N$, is large, and is order $N^2$ at reciprocal lattice points. 
\newline
\footnotetext{The term $\langle e^{-i \vec{\delta_{i}} \cdot {\vec{q}}} \rangle$ can alternatively be evaluated by expansion through a power series followed by averaging each term over the atomic displacement distribution function: $\langle e^{-i \vec{\delta_{i}} \cdot {\vec{q}}} \rangle = 1 + (-i \vec{\delta_{i}} \cdot {\vec{q}}) + \frac{(-i \vec{\delta_{i}} \cdot {\vec{q}})^2}{2!} + ... = 1 - \frac{1}{2}(\vec{\delta_i} \cdot \vec{q})^2 + ... $ . (The average over the first order term is 0.) If the displacement distribution function is Gaussian, $\langle e^{-i \vec{\delta_{i}} \cdot {\vec{q}}} \rangle = e^{-\frac{1}{2} \langle ( \vec{\delta_i} \cdot \vec{q} )^2 \rangle }$. }
\\
The `diffuse scattering' is the residual scattering described by equation 1 that cannot be accounted for by $I(\vec{q})_{Bragg}$:
\begin{equation}
\begin{aligned}
I(\vec{q})_{diffuse} & = \langle I(\vec{q})_{total} - I(\vec{q})_{Bragg} \rangle \\
& = \sum\limits_{c,c'} e^{-i \vec{q} \cdot (\vec{u_c} - \vec{u_c'})} \sum\limits_{i,j} f_i f_j e^{-i \vec{q} (\vec{r_i} - \vec{r_j})} \lbrack \langle e^{ -i \vec{q} \cdot (\vec{\delta}_{ci} - \vec{\delta}_{c'j} )} \rangle -  e^{-\frac{1}{2} (\langle (\vec{\delta}_i \cdot \vec{q})^2 \rangle + \langle (\vec{\delta}_j \cdot \vec{q})^2 \rangle) } \rbrack \\
\end{aligned}
\end{equation}
Note that:
\begin{equation}
\langle e^{-i {\vec{q}} \cdot (\vec{\delta}_{ci} - \vec{\delta}_{c'j}) } \rangle = \int e^{-\frac{1}{2} \vec{q}^\intercal V_{ci,c'j} \vec{q} - i \vec{q}^\intercal (\vec{\delta}_{ci} - \vec{\delta}_{c'j}) } d \vec{\delta} = e^{ -\frac{1}{2} \vec{q}^\intercal \lbrack V_{ci,c'i} + V_{cj,c'j} - 2V_{ci,c'j} \rbrack \vec{q} }
\end{equation}
Thus,
\begin{equation}
\begin{aligned}
I(\vec{q})_{diffuse} & = \sum\limits_{c,c'} e^{-i \vec{q} \cdot (\vec{u_c} - \vec{u_c'})} \sum\limits_{i,j} \tilde{f_i} \tilde{f_j} e^{-i \vec{q} (\vec{r_i} - \vec{r_j})} \lbrack e^{ \vec{q}^\intercal V_{ci,c'j} \vec{q} } - 1 \rbrack \\
& = \sum\limits_{c,c'} e^{-i \vec{q} \cdot (\vec{u_c} - \vec{u_c'})} \sum\limits_{i,j} f_i f_j e^{-i \vec{q} (\vec{r_i} - \vec{r_j})} e^{-\frac{1}{2} (\langle (\vec{\delta}_i \cdot \vec{q})^2 \rangle + \langle (\vec{\delta}_j \cdot \vec{q})^2 \rangle) } \lbrack e^{\langle ( \vec{\delta}_{ci} \cdot \vec{q} ) ( \vec{\delta}_{c'j} \cdot \vec{q} ) \rangle} - 1 \rbrack \\
\end{aligned}
\end{equation}
where $\tilde{f_i}$ is the atomic form factor decayed by its Debye-Waller factor, $e^{ -\frac{1}{2} \vec{q}^\intercal V_{ci,c'i} \vec{q} }$. The first version of equation 8 uses the interatomic covariance matrix, $V$, but could be equivalently written using the interatomic displacement covariance matrix, $U$, noting that $U = V_{ci,c'i} + V_{cj,c'j} - 2V_{ci,c'j} = \langle (\vec{\delta}_i - \vec{\delta}_j ^\intercal)^2 \rangle $. We can then compute $I(\vec{q})_{diffuse}$ as the total intensity modulated by this single matrix:
\begin{equation}
I(\vec{q})_{diffuse} = \sum\limits_{c,c'} e^{-i \vec{q} \cdot (\vec{u_c} - \vec{u_c'})} \sum\limits_{i,j} f_i f_j e^{-i \vec{q} (\vec{r_i} - \vec{r_j})} \lbrack e^{-\frac{1}{2}\vec{q^T}U_{ij}\vec{q}} - 1 \rbrack
\end{equation}
Assuming there is no coupling between unit cells c and c', then the covariance matrix $V$ (or displacement covariance matrix $U$) is nonzero only for $c=c'$, in which case the sum over unit cells evaluates to $N$, the number of unit cells.

\section{The form of diffuse scattering in specific disordered regimes}

\subsection{Translational disorder of the asymmetric unit} Suppose that individual protein molecules are displaced from their mean lattice positions, and that these displacement sample a Gaussian distribution. Also initially assume that the cell is P1, i.e. that there is one copy of the protein per unit cell. Then the expression for the diffuse intensity in equation 8 can be simplified by recognizing that $\forall(i,j) \, \vec{\delta}_i = \vec{\delta}_j$:
\begin{equation}
\begin{aligned}
I(\vec{q})_{diffuse} & = \sum\limits_{c,c'} e^{-i \vec{q} \cdot (\vec{u_c} - \vec{u_c'})} \sum\limits_{i,j} f_i f_j e^{-i \vec{q} (\vec{r_i} - \vec{r_j})} e^{-(\vec{\delta} \cdot \vec{q})^2} \lbrack e^{\langle ( \vec{\delta} \cdot \vec{q} )^2 } - 1 \rbrack \\
& = \sum\limits_{c,c'} e^{-i \vec{q} \cdot (\vec{u_c} - \vec{u_c'})} \sum\limits_{i,j} f_i f_j e^{-i \vec{q} (\vec{r_i} - \vec{r_j})} \lbrack 1 - e^{ - ( \vec{\delta} \cdot \vec{q} )^2 } \rbrack \\
& = N \vert F_{P1}(\vec{q}) \vert ^2 \lbrack 1 - e^{ - ( \vec{\delta} \cdot \vec{q} )^2 } \rbrack \\
\end{aligned}
\end{equation}
where $\vert F_{P1}(\vec{q}) \vert ^2$ is the unit cell structure factor and $N$ is the number of unit cells, respectively. For non-P1 cells, where individual protein molecules coincide with the asymmetric unit, equation 11 can be expressed as:
\begin{equation}
I(\vec{q})_{diffuse} = N \sum\limits_{m=1}^M \vert F_{asu}(R_m\vec{q}) \vert ^2 \lbrack 1 - e^{ - ( \vec{\delta} \cdot \vec{q} )^2 } \rbrack
\end{equation}
where $R_m$ is the m'th symmetry operator for the space group under consideration. This equation agrees with the derivation of Ayyer \textit{et. al.} \cite{pmid26863980}

\subsection{Unstrained conformational disorder} Suppose that displacements occur at the atomic rather than individual protein level, and that these displacements are uncorrelated between atoms. Here we refer to this type of disorder as `strained.' Then, the sum over atom pairs in equation 8 will only be nonzero when $i = j$:
\begin{equation}
\begin{aligned}
I(\vec{q})_{diffuse} & = \sum\limits_{c,c'} e^{-i \vec{q} \cdot (\vec{u_c} - \vec{u_c'})} \sum\limits_{i,i} f_i f_i e^{-i \vec{q} (\vec{r_i} - \vec{r_i})} e^{-\frac{1}{2} (\langle (\vec{\delta}_i \cdot \vec{q})^2 \rangle + \langle (\vec{\delta}_i \cdot \vec{q})^2 \rangle) } \lbrack e^{\langle ( \vec{\delta}_{ci} \cdot \vec{q} ) ( \vec{\delta}_{c'i} \cdot \vec{q} ) \rangle} - 1 \rbrack \\
& = N \sum f_i^2 e^{-\langle (\vec{\delta}_i \cdot \vec{q})^2 \rangle } \lbrack e^{ ( \vec{\delta}_{i} \cdot \vec{q} )^2 } - 1 \rbrack = N \sum f_i^2 \lbrack 1 - e^{- (\vec{\delta}_i \cdot \vec{q})^2 } \rbrack \\
\end{aligned}
\end{equation}
In the above equation, $\vec{\delta}_i$ is related to the anisotropic displacement parameters (ADPs). Specifically, $ \langle \vec{\delta}_i \vec{\delta}_i^\intercal \rangle $ forms the 3 x 3 symmetric tensor conventionally termed $U$, whose eigenvectors and eigenvalues correspond to the atomic displacement axes and the mean square displacements, respectively\footnotemark. The isotropic B factor equivalent can be estimated from the trace of $U$: $B_{iso} = 8\pi^2 \langle \vec{\delta}^2 \rangle = \frac{8}{3} \pi^2 tr(W) $, where $W$ is the diagonalized $U$.

\footnotetext{The six unique values of $U$ are listed in the ANISOU records of PDB files in the order $U_{11}$, $U_{22}$, $U_{33}$, $U_{12}$, $U_{13}$ , $U_{23}$. The listed values are scaled by a factor of $10^4$\cite{pmid24673391}. }

\subsection{Strained conformational disorder} Suppose that the displacements between different atoms are correlated. Here we will refer to such models of disorder as `strained.' In the anisotropic case, $I(\vec{q})_{diffuse}$ can be evaluated from equations 8 or 9. In the case that disorder is assumed to be isotropic (or the model only has sufficient resolution to describe isotropic disorder), the ($n$ x $n$ x $3$ x $3$) anisotropic displacement covariance matrix can be reduced to the ($n$ x $n$) correlation matrix, $C$, by the following relationship for the atom pair $(i, j)$:
\begin{equation}
C_{ij} = \frac{ \langle \vec{\delta}_i^\intercal \cdot \vec{\delta}_j \rangle }{\sqrt{\langle \vec{\delta}_i^\intercal \cdot \vec{\delta}_i \rangle \langle \vec{\delta}_j^\intercal \cdot \vec{\delta}_j \rangle }} =\frac{Tr(\langle \vec{\delta}_i \vec{\delta}_j^\intercal \rangle}{\sqrt{Tr(\langle \vec{\delta}_i \vec{\delta}_i^\intercal \rangle)Tr(\langle \vec{\delta}_j \vec{\delta}_j^\intercal \rangle)}} = \frac{ Tr(\langle \vec{\delta}_i \vec{\delta}_j^\intercal \rangle}{b_i^{\frac{1}{2}}b_{j}^{\frac{1}{2}}}
\end{equation}
where $b_i = \langle \vec{\delta}_i \vec{\delta}_i^\intercal \rangle$, and thus related to the Debye Waller factor: specifically, $b = \frac{B_{iso}}{8\pi^2}$, where $B_{iso}$ is the values of the conventional B factor computed during crystallographic refinement. Then, the expression for $I(\vec{q})_{diffuse}$ in equation 8 simplifies to:
\begin{equation}
I(\vec{q})_{diffuse} = \sum\limits_{i}\sum\limits_{j}f_{i}f_{j}e^{i\vec{q}(\vec{r_{i}} - \vec{r_{j}})} \lbrack e^{-\frac{1}{2}\vec{q}\,^2(b_i + b_j - 2 C_{ij}\sqrt{b_i b_{j}})} - 1 \rbrack
\end{equation}

\section{Inference of the Correlation Matrix}

Let us assume that we have many measurements of $I_\mathrm{ensemble}$ and wish to infer a model of the form just discussed. To do this, we may minimize the least squares loss function
\[
\mathcal{O} ( \{ C_{kk'} \} )= \sum_{\{\mathbf{q}\}} \left| I_\mathrm{esb}^\mathrm{obs} (\mathbf{q}) 
- I_\mathrm{esb}^\mathrm{calc} (\mathbf{q}; \{ C_{kk'} \}) \right|^2
\]
where ``obs'' and ``calc'' indicate the observed and modeled values respectively. Our task is to vary $ \{ C_{kk'} \}$ to minimize $\mathcal{O}$. 

Let $A_{k k'} (\mathbf{q}) \equiv f_{k}f_{k'} \exp \{ \, i \mathbf{q} (\mathbf{r_{k}}- \mathbf{r_{k'}}) - \frac{1}{2} q^2(B_k + B_{k'} ) \, \}$ and $\alpha_{k k'} \equiv (B_k B_{k'})^\frac{1}{2}$. Then we may write
\[
I_\mathrm{esb}^\mathrm{calc} (\mathbf{q}; \{ C_{kk'} \}) = \sum_{k k'} A_{k k'} (\mathbf{q}) \, e^{\alpha_{k k'} C_{k k'}}
\]
and the derivatives of the objective function take on a simple form
\begin{align*}
\frac{\partial \mathcal{O}} {\partial C_{kk'}} &= \sum_{\{\mathbf{q}\}}
2 \alpha_{k k'} A_{k k'} e^{\alpha_{k k'} C_{k k'}} 
\left( A_{k k'} e^{\alpha_{k k'} C_{k k'}} - I_\mathrm{esb}^\mathrm{obs}  \right) \\
%
&= \sum_{\{\mathbf{q}\}} 2 \alpha_{k k'} I_\mathrm{esb}^\mathrm{calc} \left[
I_\mathrm{esb}^\mathrm{calc} - I_\mathrm{esb}^\mathrm{obs}
\right]
\end{align*}
this shows the derivative is zero and the objective is at an extreme when $C_{kk'} = \hat{C}_{kk'}$ with
\[
\hat{C}_{kk'} = \alpha_{kk'}^{-1} 
\log \frac{ \sum_{\{\mathbf{q}\}} I_\mathrm{esb}^\mathrm{obs} (\mathbf{q}) A_{kk'} (\mathbf{q})}
{\sum_{\{\mathbf{q}\}} A_{kk'}^2 (\mathbf{q}) }
\]
computing the second derivative
\[
\frac{\partial^2 \mathcal{O}} {\partial C_{kk'}^2} = \sum_{\{\mathbf{q}\}}
2 \alpha_{k k'}^2 A_{k k'} e^{\alpha_{k k'} C_{k k'}} 
\left( 2 A_{k k'} e^{\alpha_{k k'} C_{k k'}} - I_\mathrm{esb}^\mathrm{obs} \right)
\]
and substituting $C = \hat{C}_{kk'}$
\[
\frac{\partial^2 \mathcal{O}} {\partial C_{kk'}^2} (\hat{C}_{kk'}) =  2 \alpha_{k k'}^2 
\frac{\sum_{\{\mathbf{q}\}} A_{kk'}^2 (\mathbf{q})I_\mathrm{esb}^{\mathrm{obs}\, 2}(\mathbf{q})}
{\sum_{\{\mathbf{q}\}} A_{kk'}^2 (\mathbf{q})}
\]
proves that this extremum is indeed the minimum and $\mathcal{O}$ is convex.

\bibliography{diffuse_scatter}

\end{document}


